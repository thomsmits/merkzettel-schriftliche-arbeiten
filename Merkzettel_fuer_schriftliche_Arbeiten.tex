
%% bare_conf.tex
%% V1.4
%% 2012/12/27
%% by Michael Shell
%% See:
%% http://www.michaelshell.org/
%% for current contact information.
%%
%% This is a skeleton file demonstrating the use of IEEEtran.cls
%% (requires IEEEtran.cls version 1.8 or later) with an IEEE conference paper.
%%
%% Support sites:
%% http://www.michaelshell.org/tex/ieeetran/
%% http://www.ctan.org/tex-archive/macros/latex/contrib/IEEEtran/
%% and
%% http://www.ieee.org/

%%*************************************************************************
%% Legal Notice:
%% This code is offered as-is without any warranty either expressed or
%% implied; without even the implied warranty of MERCHANTABILITY or
%% FITNESS FOR A PARTICULAR PURPOSE! 
%% User assumes all risk.
%% In no event shall IEEE or any contributor to this code be liable for
%% any damages or losses, including, but not limited to, incidental,
%% consequential, or any other damages, resulting from the use or misuse
%% of any information contained here.
%%
%% All comments are the opinions of their respective authors and are not
%% necessarily endorsed by the IEEE.
%%
%% This work is distributed under the LaTeX Project Public License (LPPL)
%% ( http://www.latex-project.org/ ) version 1.3, and may be freely used,
%% distributed and modified. A copy of the LPPL, version 1.3, is included
%% in the base LaTeX documentation of all distributions of LaTeX released
%% 2003/12/01 or later.
%% Retain all contribution notices and credits.
%% ** Modified files should be clearly indicated as such, including  **
%% ** renaming them and changing author support contact information. **
%%
%% File list of work: IEEEtran.cls, IEEEtran_HOWTO.pdf, bare_adv.tex,
%%                    bare_conf.tex, bare_jrnl.tex, bare_jrnl_compsoc.tex,
%%                    bare_jrnl_transmag.tex
%%*************************************************************************

% *** Authors should verify (and, if needed, correct) their LaTeX system  ***
% *** with the testflow diagnostic prior to trusting their LaTeX platform ***
% *** with production work. IEEE's font choices can trigger bugs that do  ***
% *** not appear when using other class files.                            ***
% The testflow support page is at:
% http://www.michaelshell.org/tex/testflow/



% Note that the a4paper option is mainly intended so that authors in
% countries using A4 can easily print to A4 and see how their papers will
% look in print - the typesetting of the document will not typically be
% affected with changes in paper size (but the bottom and side margins will).
% Use the testflow package mentioned above to verify correct handling of
% both paper sizes by the user's LaTeX system.
%
% Also note that the "draftcls" or "draftclsnofoot", not "draft", option
% should be used if it is desired that the figures are to be displayed in
% draft mode.
%
%\documentclass[journal,compsoc,final,a4paper]{IEEEtran}
\documentclass[conference,final,a4paper]{IEEEtran}
%\documentclass[journal,final,twoside,a4paper]{IEEEtran}
%\documentclass[technote,final,a4paper]{IEEEtran}
% Add the compsoc option for Computer Society conferences.
%
% If IEEEtran.cls has not been installed into the LaTeX system files,
% manually specify the path to it like:
% \documentclass[conference]{../sty/IEEEtran}





% Some very useful LaTeX packages include:
% (uncomment the ones you want to load)


% *** MISC UTILITY PACKAGES ***
%
%\usepackage{ifpdf}
% Heiko Oberdiek's ifpdf.sty is very useful if you need conditional
% compilation based on whether the output is pdf or dvi.
% usage:
% \ifpdf
%   % pdf code
% \else
%   % dvi code
% \fi
% The latest version of ifpdf.sty can be obtained from:
% http://www.ctan.org/tex-archive/macros/latex/contrib/oberdiek/
% Also, note that IEEEtran.cls V1.7 and later provides a builtin
% \ifCLASSINFOpdf conditional that works the same way.
% When switching from latex to pdflatex and vice-versa, the compiler may
% have to be run twice to clear warning/error messages.






% *** CITATION PACKAGES ***
%
\usepackage{cite}
% cite.sty was written by Donald Arseneau
% V1.6 and later of IEEEtran pre-defines the format of the cite.sty package
% \cite{} output to follow that of IEEE. Loading the cite package will
% result in citation numbers being automatically sorted and properly
% "compressed/ranged". e.g., [1], [9], [2], [7], [5], [6] without using
% cite.sty will become [1], [2], [5]--[7], [9] using cite.sty. cite.sty's
% \cite will automatically add leading space, if needed. Use cite.sty's
% noadjust option (cite.sty V3.8 and later) if you want to turn this off
% such as if a citation ever needs to be enclosed in parenthesis.
% cite.sty is already installed on most LaTeX systems. Be sure and use
% version 4.0 (2003-05-27) and later if using hyperref.sty. cite.sty does
% not currently provide for hyperlinked citations.
% The latest version can be obtained at:
% http://www.ctan.org/tex-archive/macros/latex/contrib/cite/
% The documentation is contained in the cite.sty file itself.






% *** GRAPHICS RELATED PACKAGES ***
%
\ifCLASSINFOpdf
  % \usepackage[pdftex]{graphicx}
  % declare the path(s) where your graphic files are
  % \graphicspath{{../pdf/}{../jpeg/}}
  % and their extensions so you won't have to specify these with
  % every instance of \includegraphics
  % \DeclareGraphicsExtensions{.pdf,.jpeg,.png}
\else
  % or other class option (dvipsone, dvipdf, if not using dvips). graphicx
  % will default to the driver specified in the system graphics.cfg if no
  % driver is specified.
  % \usepackage[dvips]{graphicx}
  % declare the path(s) where your graphic files are
  % \graphicspath{{../eps/}}
  % and their extensions so you won't have to specify these with
  % every instance of \includegraphics
  % \DeclareGraphicsExtensions{.eps}
\fi
% graphicx was written by David Carlisle and Sebastian Rahtz. It is
% required if you want graphics, photos, etc. graphicx.sty is already
% installed on most LaTeX systems. The latest version and documentation
% can be obtained at: 
% http://www.ctan.org/tex-archive/macros/latex/required/graphics/
% Another good source of documentation is "Using Imported Graphics in
% LaTeX2e" by Keith Reckdahl which can be found at:
% http://www.ctan.org/tex-archive/info/epslatex/
%
% latex, and pdflatex in dvi mode, support graphics in encapsulated
% postscript (.eps) format. pdflatex in pdf mode supports graphics
% in .pdf, .jpeg, .png and .mps (metapost) formats. Users should ensure
% that all non-photo figures use a vector format (.eps, .pdf, .mps) and
% not a bitmapped formats (.jpeg, .png). IEEE frowns on bitmapped formats
% which can result in "jaggedy"/blurry rendering of lines and letters as
% well as large increases in file sizes.
%
% You can find documentation about the pdfTeX application at:
% http://www.tug.org/applications/pdftex




% *** MATH PACKAGES ***
%
%\usepackage[cmex10]{amsmath}
% A popular package from the American Mathematical Society that provides
% many useful and powerful commands for dealing with mathematics. If using
% it, be sure to load this package with the cmex10 option to ensure that
% only type 1 fonts will utilized at all point sizes. Without this option,
% it is possible that some math symbols, particularly those within
% footnotes, will be rendered in bitmap form which will result in a
% document that can not be IEEE Xplore compliant!
%
% Also, note that the amsmath package sets \interdisplaylinepenalty to 10000
% thus preventing page breaks from occurring within multiline equations. Use:
%\interdisplaylinepenalty=2500
% after loading amsmath to restore such page breaks as IEEEtran.cls normally
% does. amsmath.sty is already installed on most LaTeX systems. The latest
% version and documentation can be obtained at:
% http://www.ctan.org/tex-archive/macros/latex/required/amslatex/math/





% *** SPECIALIZED LIST PACKAGES ***
%
%\usepackage{algorithmic}
% algorithmic.sty was written by Peter Williams and Rogerio Brito.
% This package provides an algorithmic environment fo describing algorithms.
% You can use the algorithmic environment in-text or within a figure
% environment to provide for a floating algorithm. Do NOT use the algorithm
% floating environment provided by algorithm.sty (by the same authors) or
% algorithm2e.sty (by Christophe Fiorio) as IEEE does not use dedicated
% algorithm float types and packages that provide these will not provide
% correct IEEE style captions. The latest version and documentation of
% algorithmic.sty can be obtained at:
% http://www.ctan.org/tex-archive/macros/latex/contrib/algorithms/
% There is also a support site at:
% http://algorithms.berlios.de/index.html
% Also of interest may be the (relatively newer and more customizable)
% algorithmicx.sty package by Szasz Janos:
% http://www.ctan.org/tex-archive/macros/latex/contrib/algorithmicx/




% *** ALIGNMENT PACKAGES ***
%
%\usepackage{array}
% Frank Mittelbach's and David Carlisle's array.sty patches and improves
% the standard LaTeX2e array and tabular environments to provide better
% appearance and additional user controls. As the default LaTeX2e table
% generation code is lacking to the point of almost being broken with
% respect to the quality of the end results, all users are strongly
% advised to use an enhanced (at the very least that provided by array.sty)
% set of table tools. array.sty is already installed on most systems. The
% latest version and documentation can be obtained at:
% http://www.ctan.org/tex-archive/macros/latex/required/tools/


% IEEEtran contains the IEEEeqnarray family of commands that can be used to
% generate multiline equations as well as matrices, tables, etc., of high
% quality.




% *** SUBFIGURE PACKAGES ***
%\ifCLASSOPTIONcompsoc
%  \usepackage[caption=false,font=normalsize,labelfont=sf,textfont=sf]{subfig}
%\else
%  \usepackage[caption=false,font=footnotesize]{subfig}
%\fi
% subfig.sty, written by Steven Douglas Cochran, is the modern replacement
% for subfigure.sty, the latter of which is no longer maintained and is
% incompatible with some LaTeX packages including fixltx2e. However,
% subfig.sty requires and automatically loads Axel Sommerfeldt's caption.sty
% which will override IEEEtran.cls' handling of captions and this will result
% in non-IEEE style figure/table captions. To prevent this problem, be sure
% and invoke subfig.sty's "caption=false" package option (available since
% subfig.sty version 1.3, 2005/06/28) as this is will preserve IEEEtran.cls
% handling of captions.
% Note that the Computer Society format requires a larger sans serif font
% than the serif footnote size font used in traditional IEEE formatting
% and thus the need to invoke different subfig.sty package options depending
% on whether compsoc mode has been enabled.
%
% The latest version and documentation of subfig.sty can be obtained at:
% http://www.ctan.org/tex-archive/macros/latex/contrib/subfig/




% *** FLOAT PACKAGES ***
%
%\usepackage{fixltx2e}
% fixltx2e, the successor to the earlier fix2col.sty, was written by
% Frank Mittelbach and David Carlisle. This package corrects a few problems
% in the LaTeX2e kernel, the most notable of which is that in current
% LaTeX2e releases, the ordering of single and double column floats is not
% guaranteed to be preserved. Thus, an unpatched LaTeX2e can allow a
% single column figure to be placed prior to an earlier double column
% figure. The latest version and documentation can be found at:
% http://www.ctan.org/tex-archive/macros/latex/base/


%\usepackage{stfloats}
% stfloats.sty was written by Sigitas Tolusis. This package gives LaTeX2e
% the ability to do double column floats at the bottom of the page as well
% as the top. (e.g., "\begin{figure*}[!b]" is not normally possible in
% LaTeX2e). It also provides a command:
%\fnbelowfloat
% to enable the placement of footnotes below bottom floats (the standard
% LaTeX2e kernel puts them above bottom floats). This is an invasive package
% which rewrites many portions of the LaTeX2e float routines. It may not work
% with other packages that modify the LaTeX2e float routines. The latest
% version and documentation can be obtained at:
% http://www.ctan.org/tex-archive/macros/latex/contrib/sttools/
% Do not use the stfloats baselinefloat ability as IEEE does not allow
% \baselineskip to stretch. Authors submitting work to the IEEE should note
% that IEEE rarely uses double column equations and that authors should try
% to avoid such use. Do not be tempted to use the cuted.sty or midfloat.sty
% packages (also by Sigitas Tolusis) as IEEE does not format its papers in
% such ways.
% Do not attempt to use stfloats with fixltx2e as they are incompatible.
% Instead, use Morten Hogholm'a dblfloatfix which combines the features
% of both fixltx2e and stfloats:
%
% \usepackage{dblfloatfix}
% The latest version can be found at:
% http://www.ctan.org/tex-archive/macros/latex/contrib/dblfloatfix/




% *** PDF, URL AND HYPERLINK PACKAGES ***
%
\usepackage{url}
% url.sty was written by Donald Arseneau. It provides better support for
% handling and breaking URLs. url.sty is already installed on most LaTeX
% systems. The latest version and documentation can be obtained at:
% http://www.ctan.org/tex-archive/macros/latex/contrib/url/
% Basically, \url{my_url_here}.

\usepackage{color}
\definecolor{darkblue}{rgb}{0,0,.5}

\usepackage[utf8]{inputenx}
\usepackage[ngerman]{betababel}

\ifCLASSINFOpdf
   \usepackage[
	    unicode=true,
      hypertexnames=false,
      colorlinks=true,
%      colorlinks=false,
      linkcolor=darkblue,
      citecolor=darkblue,
      urlcolor=darkblue,
      pdftex
   ]{hyperref}
%	 \PrerenderUnicode{ü}
	 \hypersetup{
	    pdftitle={Merkzettel für Studien- und Abschlussarbeiten},
			pdfauthor={Prof. Dr. Jörn Fischer, Prof. Thomas Smits, Prof. Dr. Thomas Ihme},
			pdfdisplaydoctitle=true
			}
\else
   \usepackage[hypertex]{hyperref}
\fi



% *** Do not adjust lengths that control margins, column widths, etc. ***
% *** Do not use packages that alter fonts (such as pslatex).         ***
% There should be no need to do such things with IEEEtran.cls V1.6 and later.
% (Unless specifically asked to do so by the journal or conference you plan
% to submit to, of course. )




% correct bad hyphenation here
\hyphenation{For-schungs-fra-ge}


\begin{document}
%
% paper title
% can use linebreaks \\ within to get better formatting as desired
% Do not put math or special symbols in the title.
\title{Merkzettel für Studien- und Abschlussarbeiten}
%\markboth{Erster Text}{Zweiter text}
%\IEEEpubid{(c) Hochschule Mannheim, 2013}

% author names and affiliations
% use a multiple column layout for up to three different
% affiliations
\author{\IEEEauthorblockN{Prof. Dr. Jörn Fischer\IEEEauthorrefmark{1},
Prof. Thomas Smits\IEEEauthorrefmark{2},
Prof. Dr. Thomas Ihme\IEEEauthorrefmark{3}}
\IEEEauthorblockA{Hochschule Mannheim\\
Paul-Wittsack-Str. 10, 
68163 Mannheim\\
\IEEEauthorrefmark{1}j.fischer@hs-mannheim.de
\IEEEauthorrefmark{2}t.smits@hs-mannheim.de\\
\IEEEauthorrefmark{3}t.ihme@hs-mannheim.de}
%\and
%\IEEEauthorblockN{Homer Simpson}
%\IEEEauthorblockA{Twentieth Century Fox\\
%Springfield, USA\\
%Email: homer@thesimpsons.com}
%\and
%\IEEEauthorblockN{James Kirk\\ and Montgomery Scott}
%\IEEEauthorblockA{Starfleet Academy\\
%San Francisco, California 96678-2391\\
%Telephone: (800) 555--1212\\
%Fax: (888) 555--1212}
}

% conference papers do not typically use \thanks and this command
% is locked out in conference mode. If really needed, such as for
% the acknowledgment of grants, issue a \IEEEoverridecommandlockouts
% after \documentclass

% for over three affiliations, or if they all won't fit within the width
% of the page, use this alternative format:
% 
%\author{\IEEEauthorblockN{Michael Shell\IEEEauthorrefmark{1},
%Homer Simpson\IEEEauthorrefmark{2},
%James Kirk\IEEEauthorrefmark{3}, 
%Montgomery Scott\IEEEauthorrefmark{3} and
%Eldon Tyrell\IEEEauthorrefmark{4}}
%\IEEEauthorblockA{\IEEEauthorrefmark{1}School of Electrical and Computer Engineering\\
%Georgia Institute of Technology,
%Atlanta, Georgia 30332--0250\\ Email: see http://www.michaelshell.org/contact.html}
%\IEEEauthorblockA{\IEEEauthorrefmark{2}Twentieth Century Fox, Springfield, USA\\
%Email: homer@thesimpsons.com}
%\IEEEauthorblockA{\IEEEauthorrefmark{3}Starfleet Academy, San Francisco, California 96678-2391\\
%Telephone: (800) 555--1212, Fax: (888) 555--1212}
%\IEEEauthorblockA{\IEEEauthorrefmark{4}Tyrell Inc., 123 Replicant Street, Los Angeles, California 90210--4321}}




% use for special paper notices
%\IEEEspecialpapernotice{(Invited Paper)}




% make the title area
\maketitle

% As a general rule, do not put math, special symbols or citations
% in the abstract
\begin{abstract}
Dieses Dokument ist eine Zusammenfassung der wichtigsten Aspekte, die beim Verfassen einer wissenschaftlichen Arbeit zu beachten sind. Die hier beschriebenen Inhalte erheben weder den Anspruch auf Vollständigkeit, noch für Arbeiten bei anderen Professoren relevant zu sein. Sie sind vielmehr Orientierungshilfen, die dem Studierenden die formalen und inhaltlichen Kriterien für die Bewertung ihrer Arbeit in aller Kürze darlegen. 
\end{abstract}

% no keywords




% For peer review papers, you can put extra information on the cover
% page as needed:
% \ifCLASSOPTIONpeerreview
% \begin{center} \bfseries EDICS Category: 3-BBND \end{center}
% \fi
%
% For peerreview papers, this IEEEtran command inserts a page break and
% creates the second title. It will be ignored for other modes.
\IEEEpeerreviewmaketitle



\section{Einleitung}

Es gibt zahlreiche Abhandlungen über das Schreiben wissenschaftlicher Arbeiten \cite{kraemer, kornmeier, stengel, kara, eco}. Die meisten davon gehen viel mehr ins Detail als dieser \emph{Merkzettel}. Viele der hier aufgeführten Aspekte schriftlicher Arbeiten sind auch bereits im \emph{Leitfaden für schriftliche Arbeiten} von Herrn Prof. Thomas Smits ausführlicher behandelt. Das Dokument ist lesenswert und leicht auf seiner Homepage (\url{http://www.smits-net.de/arbeiten.html}) zu finden.

\section{Form, Stil, Sprache und Umfang}
Praxissemesterberichte, Studien-, Bachelor- und Masterarbeiten sollten einspaltig sein. Benutzen Sie bitte eine für Fließtext geeignete Schriftart (z.\,B. Times, Palatino) in der Größe 10, 11 oder 12 Punkt. Fußnoten werden entsprechend 1-2 Punkt kleiner gesetzt. Lassen Sie vor allem am linken und rechten Rand mindestens 3 cm Platz. Ein Zeilenabstand von 1,1-fach oder 1,2-fach ist meist empfehlenswert. Falls Sie Textteile hervorheben möchten, dann setzen Sie sie kursiv (nicht fett und auch nicht unterstrichen). Nummerieren Sie die Kapitel beginnend bei 1 hierarchisch mit arabischen Zahlen (1.1, 1.2 etc.). Für Studien- und Abschlussarbeiten existieren an der Fakultät geeignete Vorlagen, die Sie benutzen können.

Schreiben Sie die Arbeit so, dass ein fachkundiger Dritter in der Lage ist, den Text zu verstehen und die darin enthaltenen Schlüsse nachvollziehen zu können. Hierzu sollten alle nicht bekannten Fakten mit Literaturstellen belegt werden. Schreiben Sie in einer einfachen, gut verständlichen Sprache mit kurzen Sätzen. Schreiben sie durchgängig in der Gegenwartsform und im passiv (z.\,B. "` \dots wird untersucht"', "` \dots zeigt folgende Ergebnisse"'). Nutzen Sie wenn möglich deutsche Begriffe, auf keinen Fall jedoch Mischformen wie \emph{downgeloaded} oder \emph{upgedatet}. Abkürzungen müssen in einem Abkürzungsverzeichnis aufgeführt und bei der ersten Verwendung auch ausgeschrieben werden. Durchschnittswerte für die Länge (ohne Anhang) eines Praxissemesterberichts (Projektbericht) sind 10-15 Seiten, einer Bachelorarbeit sind 50-60 Seiten und einer Masterarbeit sind 60-80 Seiten.

\section{Aufbau}

Ihre Arbeit sollte mindestens ein Titelblatt, ein Inhaltsverzeichnis, einen Textteil und ein Literaturverzeichnis beinhalten. Der Titel ist auch in englischer Sprache anzugeben. Zusätzlich kann die Arbeit noch ein Abbildungsverzeichnis, ein Quellenverzeichnis, ein Abkürzungsverzeichnis, ein Tabellenverzeichnis, ein Symbolverzeichnis und einen Anhang enthalten. Alle Abbildungen, Formeln, Tabellen und Listings im Text sind gemäß Kapitelnummerierungen durchzunummerieren und mit ihrer jeweiligen Nummer in den zugehörigen Verzeichnissen aufzulisten sowie im Fließtext zu behandeln.

Ein guter Stil ist es, am Anfang jedes Kapitels einen einleitenden Überblick zu geben und das Kapitel mit einer Zusammenfassung und mit einer Überleitung ins nächste Kapitel abzuschließen. 

Die einzelnen Bestandteile der Arbeit werden im folgenden näher erläutert, wobei die Abschnittsbezeichnungen keine Kapitelüberschriften sein sollen. Die endgültige Struktur orientiert sich am Thema und der Vorgehensweise.

%\begin{center}\"Uberblick\end{center}
\begin{itemize}
\item \emph{Überblick (engl. Abstract):} Fasst die Arbeit sehr kurz zusammen ($<20$ Zeilen). Es wird kurz der Hintergrund und das Problem erläutert sowie die Lösung und das Ergebnis vorgestellt. Ein zusätzlicher englischer Abstract hilft auch auswärtigen Lesern, einen Einblick zu bekommen, und sollte nicht fehlen.
\item \emph{Einleitung:} Beschreibt den Hintergrund der Arbeit, das bearbeitete Problem und die Untersuchungsmethoden. Am Ende wird kurz der Aufbau der Arbeit erläutert.
\item \emph{Grundlegend Definitionen:} Beschreibt die bekannten Methoden, die genutzt werden, um zur Lösung zu kommen. Die Vorgehensweise sollte mit Literaturquellen belegt werden.
\item \emph{Verwandte Arbeiten:} Hier werden Arbeiten von anderen Forschern, die das gleiche oder ein ähnliches Gebiet bearbeiten, erörtert. Unterschiede zu ihrem Ansatz werden herausgearbeitet. Dieses Kapitel kann auch mit dem vorherigen Kapitel verschmelzen.
\item \emph{Problem und Anforderungsanalyse:} Falls das Einleitungskapitel für die Problembeschreibung noch nicht ausgereicht hat, kann hier das Problem noch einmal detailliert analysiert werden. Die Anforderungen (Requirements) an die zu entwickelnde Lösung werden dargestellt.
\item \emph{Konzept / Lösungsansatz}: Der formale Lösungsansatz bzw. das Lösungskonzept wird hier unabhängig von seiner Implementierung beschrieben und begründet (z.\,B. Architektur etc.). Hier werden auch die Methoden dargestellt, die für zum Finden der Lösung ausgewählt werden.
\item \emph{Implementierung:} Hier können verschiedene Aspekte der Implementierung anhand eines Anwendungsbeispiels beschrieben und begründet werden.
\item \emph{Validierung:} Die Qualität des Lösungsansatzes und der Implementierung werden untersucht. Dies kann durch Experimente oder Auswertung anderer Ergebnisse geschehen.
\item \emph{Diskussion, Bewertung:} Wenn nicht bereits im Kapitel zur Validierung geschene, können hier die Ergebnisse der Arbeit noch einmal diskutiert und mit Alternativen verglichen werden.
\item \emph{Zusammenfassung und Ausblick:} Die Ergebnisse werden noch einmal zusammengefasst und ein Ausblick über mögliche Erweiterungen gegeben.
\end{itemize}


\noindent Alle logischen Schlüsse im Text sollten überlegt und überprüft werden. In \cite{stengel} werden logische Fehlschlüsse ausführlich erläutert.

\section{Zitate und Quellenangaben}
Alle von anderen Autoren gewonnenen Erkenntnisse müssen mit Quellen belegt werden. Falls Sie wörtlich zitieren werden, so muss der Text originalgetreu in Anführungszeichen wiedergegeben werden. Wird ein Teil des Textes ausgelassen, so werden Punkte in eckigen Klammern [\,\dots] an die Stelle der Auslassung gesetzt. Zusätze innerhalb des zitierten Textes bedürfen eckiger Klammern []. Gehen Sie sparsam mit wörtlichen Zitaten um. Längere wörtliche Zitate werden i.\,a. eingerückt.

Die Zitierweise muss im gesamten Text einheitlich sein. Empfehlenswert ist beispielsweise die Harvard-Zitierweise oder die in diesem Dokument verwendete numerische Zitierweise \cite{kornmeier}.

Wichtig ist, dass das Übernehmen von fremden Textstellen ohne entsprechende Kennzeichnung der Herkunft in einer wissenschaftlichen Arbeit nicht akzeptabel ist. Plagiate werden mit der Note 5.0 bewertet.

Im Allgemeinen sind Internetquellen nicht zitierfähig, da sie oft weder dem wissenschaftlichen Anspruch genügen, noch dauerhaft zur Verfügung stehen. Falls trotzdem eine Quelle aus dem Internet zitiert werden muss, z.\,B. weil weder Bücher noch Zeitschriften die Information liefern, so muss ein Ausdruck der Webseite mit Datum und Internetadresse im Anhang angehängt oder als PDF auf CD beigelegt werden.

\section{Abbildungen}
Abbildungen sind oft sehr hilfreich, um Zusammenhänge zu verdeutlichen. Soweit möglich, sollten die Abbildungen als Vektorgrafiken eingebunden werden. Die in der Abbildung enthaltenen Schriftarten sollten nach Möglichkeit die gleichen sein, wie im restlichen Dokument. Alle Beschriftungen innerhalb der Grafik sollten gut lesbar sein.
Graphen können farbig sein, wenn es der Lesbarkeit dient. Es sollte jedoch darauf geachtet werden, dass auch eine schwarz\-weiß-Kopie noch alle nötigen Informationen enthält (d.h. Liniendiagramme mit verschiedenen Linienarten z.\,B. Strichpunkt).  
Falls eine Abbildung nur in Form einer Bitmap realisiert werden kann (e,\.B. Screenshot), sollte die Auflösung mindestens 600 dpi betragen und die Qualität nicht durch Komprimierung (z.\,B. jpg) verschlechtert werden. Unkomprimierte oder verlustfrei komprimierte Bildformate wie \emph{png} oder \emph{bmp} sind zu bevorzugen. Nur für echte Fotos ist \emph{jpg} geeignet.

Jede Abbildung und Tabelle sollte im Fließtext referenziert und behandelt werden. Die Beschriftung unter den Abbildungen sollte die Abbildung vollständig und verständlich beschreiben, auch ohne dass man den restlichen Text gelesen hat. Gleiches gilt für die Beschriftungen von Tabellen, die über die Tabellen gesetzt werden.

\section{Bewertung}
Die Bewertung im konkreten Fall hängt immer vom jeweiligen Prüfer, dem Fachgebiet und dem gewählten Thema ab. Als Orientierung kann man allein für die Bewertung der Abschlussarbeit in etwa folgende Gewichtung zugrunde legen:
\begin{itemize}
\item 70\% - Inhalt (Qualität und die Relevanz des Themas, der Forschungsfrage und der zitierten Literatur, Aufbau der Arbeit  etc.)
\item 30\% - Stil und Form (korrekte Verwendung von Wörtern, wissenschaftlichen Ausdrücken, Sprachlogik, Ästhetik, Rechtschreibung etc.)
\end{itemize}

\section{Anmerkungen}
Zur Qualitätssicherung Ihrer Arbeit ist u.\,a. folgende Vorgehensweise hilfreich:
\begin{itemize}
\item Wenn es irgendwie möglich ist, sollten Sie die Arbeit auch Kommilitonen lesen lassen. Selbst Verwandte, Freunde und Bekannte, die nicht \emph{vom Fach} sind, finden vielleicht Fehler oder kommentieren Ihre Arbeit.
\item Um sicher zu stellen, dass Sie die hier beschriebenen Aspekte beachten, sollten Sie die Formalien z.\,B. nach jedem geschriebenen Kapitel überprüfen.
\item Lassen Sie sich von Ihrem betreuenden Professor alle Randbedingungen und Bewertungsschemata geben und fragen Sie, worauf sie achten sollen. 
\end{itemize}



% An example of a floating figure using the graphicx package.
% Note that \label must occur AFTER (or within) \caption.
% For figures, \caption should occur after the \includegraphics.
% Note that IEEEtran v1.7 and later has special internal code that
% is designed to preserve the operation of \label within \caption
% even when the captionsoff option is in effect. However, because
% of issues like this, it may be the safest practice to put all your
% \label just after \caption rather than within \caption{}.
%
% Reminder: the "draftcls" or "draftclsnofoot", not "draft", class
% option should be used if it is desired that the figures are to be
% displayed while in draft mode.
%
%\begin{figure}[!t]
%\centering
%\includegraphics[width=2.5in]{myfigure}
% where an .eps filename suffix will be assumed under latex, 
% and a .pdf suffix will be assumed for pdflatex; or what has been declared
% via \DeclareGraphicsExtensions.
%\caption{Simulation Results.}
%\label{fig_sim}
%\end{figure}

% Note that IEEE typically puts floats only at the top, even when this
% results in a large percentage of a column being occupied by floats.


% An example of a double column floating figure using two subfigures.
% (The subfig.sty package must be loaded for this to work.)
% The subfigure \label commands are set within each subfloat command,
% and the \label for the overall figure must come after \caption.
% \hfil is used as a separator to get equal spacing.
% Watch out that the combined width of all the subfigures on a 
% line do not exceed the text width or a line break will occur.
%
%\begin{figure*}[!t]
%\centering
%\subfloat[Case I]{\includegraphics[width=2.5in]{box}%
%\label{fig_first_case}}
%\hfil
%\subfloat[Case II]{\includegraphics[width=2.5in]{box}%
%\label{fig_second_case}}
%\caption{Simulation results.}
%\label{fig_sim}
%\end{figure*}
%
% Note that often IEEE papers with subfigures do not employ subfigure
% captions (using the optional argument to \subfloat[]), but instead will
% reference/describe all of them (a), (b), etc., within the main caption.


% An example of a floating table. Note that, for IEEE style tables, the 
% \caption command should come BEFORE the table. Table text will default to
% \footnotesize as IEEE normally uses this smaller font for tables.
% The \label must come after \caption as always.
%
%\begin{table}[!t]
%% increase table row spacing, adjust to taste
%\renewcommand{\arraystretch}{1.3}
% if using array.sty, it might be a good idea to tweak the value of
% \extrarowheight as needed to properly center the text within the cells
%\caption{An Example of a Table}
%\label{table_example}
%\centering
%% Some packages, such as MDW tools, offer better commands for making tables
%% than the plain LaTeX2e tabular which is used here.
%\begin{tabular}{|c||c|}
%\hline
%One & Two\\
%\hline
%Three & Four\\
%\hline
%\end{tabular}
%\end{table}


% Note that IEEE does not put floats in the very first column - or typically
% anywhere on the first page for that matter. Also, in-text middle ("here")
% positioning is not used. Most IEEE journals/conferences use top floats
% exclusively. Note that, LaTeX2e, unlike IEEE journals/conferences, places
% footnotes above bottom floats. This can be corrected via the \fnbelowfloat
% command of the stfloats package.



%\section{Conclusion}
%The conclusion goes here.



% conference papers do not normally have an appendix


% use section* for acknowledgement
%\section*{Acknowledgment}

%The authors would like to thank...





% trigger a \newpage just before the given reference
% number - used to balance the columns on the last page
% adjust value as needed - may need to be readjusted if
% the document is modified later
%\IEEEtriggeratref{8}
% The "triggered" command can be changed if desired:
%\IEEEtriggercmd{\enlargethispage{-5in}}

% references section

% can use a bibliography generated by BibTeX as a .bbl file
% BibTeX documentation can be easily obtained at:
% http://www.ctan.org/tex-archive/biblio/bibtex/contrib/doc/
% The IEEEtran BibTeX style support page is at:
% http://www.michaelshell.org/tex/ieeetran/bibtex/
%\bibliographystyle{IEEEtran}
%\bibliographystyle{IEEEtran-de}
\bibliographystyle{plaindin}
% argument is your BibTeX string definitions and bibliography database(s)
%\bibliography{IEEEabrv,../bib/paper}
\bibliography{Literatur_Abschlussarbeiten}
%
% <OR> manually copy in the resultant .bbl file
% set second argument of \begin to the number of references
% (used to reserve space for the reference number labels box)


% that's all folks
\end{document}


